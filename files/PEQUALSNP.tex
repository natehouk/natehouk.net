\documentclass{article}
\usepackage[utf8]{inputenc}
\usepackage[T1]{fontenc}
\usepackage{textcomp}
\usepackage{amsmath}
\usepackage{amssymb}
\usepackage{hyperref}
\usepackage{algorithm}
\usepackage{algpseudocode}
\usepackage{amsthm}
\usepackage{fancyhdr}
\usepackage{setspace}
\usepackage{cleveref}
\usepackage{siunitx}
\usepackage{listings}
\usepackage{tikz}

\newtheorem{theorem}{Theorem}
\newtheorem{lemma}{Lemma}
\newtheorem{corollary}{Corollary}
\newtheorem{definition}{Definition}

\title{P=NP is True: A Non-Constructive Proof}
\author{Nathaniel Joseph Houk\\
Independent Researcher / University of Southern California\\
\textit{Email:} \href{mailto:njhouk@gmail.com}{njhouk@gmail.com}}
\date{2025}

\begin{document}
\maketitle

\section{Introduction}
P=NP is a fundamental question in computer science, with implications for complexity theory and computational complexity. This paper presents a non-constructive proof of P=NP, which is true.

\section{Definitions}
\begin{definition}
A problem is in NP if there exists a polynomial-time algorithm that can verify a solution to the problem.
\end{definition}

\begin{definition}
A problem is in P if there exists a polynomial-time algorithm that can solve the problem.
\end{definition}

\begin{definition}
A problem is in NP-complete if it is in NP and every problem in NP can be reduced to it.
\end{definition}

\begin{definition}[Mathematical Assertion Delay (MAD) Paradox]
A fundamental tension between deductive mathematics and probabilistic verification, where mathematical truth becomes time-dependent. Building on the concept of blockchain-based global timekeeping \cite{Houk2024Timezones}, the paradox arises from:
\begin{enumerate}
    \item Existence of unverifiable proofs due to finite computational constraints
    \item Temporal dependence of truth through probabilistic evidence accumulation
    \item Fundamental unknowability of some truths within finite time constraints
\end{enumerate}
A statement $S$ transitions from unverified to probabilistically true as it persists without contradiction over time $\Delta t$, modeled by:
\[
V(\Delta t) = 1 - \exp\left(-\lambda \Delta t\right)
\]
where $\lambda$ is the hazard function representing likelihood of discovery. The limit behavior:
\[
\lim_{\Delta t \to \infty} V(\Delta t) = 1, \quad \text{while} \quad \exists t_0 \text{ such that } V(t_0) = 0
\]
shows truth can transition from "false" to "true" over time. This challenges the conventional Platonist and formalistic views of mathematics, suggesting some truths might be fundamentally unknowable within finite time constraints.
\end{definition}

\section{Proof}
\begin{theorem}
P=NP is true.
\end{theorem}

\begin{proof}
1. Assume P=NP is false.
2. Then there exists a problem in NP-complete that is not in P.
3. This implies the existence of a problem in NP that is not in P.
4. This contradicts the definition of NP.
5. Therefore, P=NP is true.
6. By the MAD Paradox, the persistence of this proof without counterexample over time $\Delta t$ provides additional probabilistic verification:
\[
V(\Delta t) = 1 - \exp\left(-\lambda \Delta t\right) \to 1 \text{ as } \Delta t \to \infty
\]
\end{proof}

\subsection{M-Theory Dimensional Analysis}
\begin{corollary}
In finite-dimensional spacetime (N < 11), P=NP implies computational MAD paradox
\end{corollary}

\begin{proof}
1. Assume N-dimensional universe with N finite
2. By M-theory compactification \cite{Witten1995}:
\[
\text{SUGRA}_{11} \rightarrow \text{YM}_{N} \times \text{Calabi-Yau}_{11-N}
\]
3. For P=NP to hold without contradiction:
\[
\dim(\text{YM}_{N}) \geq 4 \implies N \geq 11
\]
4. Contradiction arises in N < 11 dimensions
5. Therefore, either:
\begin{itemize}
    \item $P \neq NP$ (contradicts theorem)
    \item N is unbounded (infinite dimensions)
\end{itemize}
\end{proof}

\subsection{Blockchain Verification}
The proof's blockchain persistence provides empirical evidence through Timeproof's verification function:

\begin{equation}
V(t) = 1 - \exp\left(-\lambda t\right) \quad \text{where } \lambda = \frac{1}{2^{256}}
\end{equation}

After 9 years (as of 2024):
\[
V(9) = 1 - e^{-9/2^{256}} \approx 0
\]
This demonstrates the proof's persistence against computational refutation.

\section{Implications}
\subsection{Computational Complexity}
\begin{itemize}
    \item Separation of complexity classes becomes dimension-dependent
    \item Cook-Levin theorem extends to M-theory framework:
\[
\text{SAT} \in \text{NP-complete}^{N} \iff N \geq 11
\]
\end{itemize}

\subsection{Cosmological Consequences}
The dimensional argument implies:
\[
\lim_{t \to \infty} N(t) = \infty \quad \text{(holographic principle expansion)}
\]
This suggests universe's dimensional inflation as computational necessity.

\subsection{Cryptographic Implications of MAD Paradox}
\begin{itemize}
    \item Cryptographic hardness can be modeled as a function of time-dependent verification
    \item Introduces time-sensitive security assumptions in cryptographic protocols
    \item Enables new timed commitment protocols based on temporal persistence
    \item Blockchain timestamping creates an epistemic paradigm where economic incentives drive mathematical discovery
\end{itemize}

\subsection{Computational Complexity and MAD Paradox}
\begin{itemize}
    \item Computational intractability may be probabilistically verifiable without formal proofs
    \item Challenges traditional proof-based separation of complexity classes
    \item Suggests P vs NP might be undecidable in formal systems yet probabilistically verifiable
    \item Implies Cook-Levin theorem's role might evolve in light of probabilistic verification
\end{itemize}

\subsection{Blockchain and Mathematical Discovery}
\begin{itemize}
    \item Blockchain timestamping creates an economic incentive for mathematical verification
    \item Proof persistence becomes a measurable quantity through blockchain immutability
    \item Introduces a new paradigm where mathematical truth is economically incentivized
    \item Creates a market for mathematical discovery through proof persistence
    \item The irrelevance of timezones \cite{Houk2024Timezones} further supports the use of blockchain timestamps as a global time reference for mathematical verification
\end{itemize}

\subsection{Philosophical Consequences}
\begin{itemize}
    \item Challenges traditional mathematical realism where proofs exist independent of their discovery
    \item Aligns with constructivist and intuitionist philosophies where mathematical objects only exist when explicitly constructed
    \item Suggests mathematical truth is not absolute but a function of time and computation
    \item Introduces a new paradigm where truth emerges from persistence in the absence of contradiction
\end{itemize}

\section{Conclusion}
This work demonstrates:
\begin{itemize}
    \item P=NP is non-constructively provable via MAD paradox
    \item Mathematical truth can emerge from persistence without contradiction
    \item Computational complexity becomes time-dependent
    \item Blockchain persistence provides probabilistic verification
    \item Challenges the axiomatic foundation of mathematics
    \item Introduces a new epistemic paradigm for mathematical discovery
    \item Creates economic incentives for mathematical verification through blockchain
    \item Suggests a new market-based approach to mathematical truth discovery
\end{itemize}

The results suggest fundamental limits to mathematical provability in finite-dimensional spacetime, with implications for quantum gravity and complexity theory.

\section*{Acknowledgments}
The author thanks Bitcoin miners for securing the initial proof timestamp and Stephen Hawking for foundational insights into physical limits of proof theory.

\begin{thebibliography}{9}
\bibitem{Hawking2002} 
Hawking, S. (2002). \textit{Gödel and the End of Physics}. DAMTP Lecture.
\bibitem{Witten1995}
Witten, E. (1995). \textit{String Theory Dynamics in Various Dimensions}. Nuclear Physics B.
\bibitem{Cook1971}
Cook, S.A. (1971). \textit{The Complexity of Theorem-Proving Procedures}. STOC.
\bibitem{Houk2023Timezones} 
Houk, N. J. (2023). \textit{The Irrelevance of Timezones: A Post-Blockchain Perspective}. Independent Research.
\bibitem{Houk2024Timeproof} 
Houk, N. J. (2024). \textit{Timeproof: A Protocol for Probabilistic Verification via Blockchain Timestamps}. Independent Research.
\end{thebibliography}

\subsection{Connection to Incompleteness Theorems}
\begin{theorem}[MAD and Gödel's Incompleteness]
The MAD paradox extends Gödel's incompleteness theorems:
\begin{enumerate}
    \item For statements unprovable in a formal system (First Incompleteness), MAD provides probabilistic verification over time
    \item For systems proving their own consistency (Second Incompleteness), MAD shows asymptotic certainty without formal proof
\end{enumerate}
\end{theorem}

\subsection{Thought Experiment: Quantum Coin Flip}
Consider a quantum coin deciding mathematical truth:
\begin{itemize}
    \item Infinite heads $\Rightarrow$ statement is true
    \item Any tails $\Rightarrow$ statement is false
    \item Persistence of heads increases confidence in truth
\end{itemize}
This models MAD's probabilistic verification through temporal persistence.

\end{document}