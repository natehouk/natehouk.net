\documentclass[12pt]{article}
\usepackage[utf8]{inputenc}
\usepackage[T1]{fontenc}
\usepackage{textcomp}
\usepackage{amsmath}
\usepackage{amssymb}
\usepackage{amsthm}
\usepackage{hyperref}
\usepackage{algorithm}
\usepackage{algpseudocode}
\usepackage{fancyhdr}
\usepackage{setspace}
\usepackage{cleveref}
\usepackage{siunitx}
\usepackage{listings}
\usepackage{tikz}
\usepackage{graphicx}
\onehalfspacing

% Define custom commands for metadata
\newcommand{\email}[1]{\hypersetup{pdfauthor={#1}}}
\newcommand{\keywords}[1]{\hypersetup{pdfkeywords={#1}}}

% Set metadata
\keywords{MAD paradox, mathematical truth, temporal verification, oracle competition, phase transitions, complexity theory, P vs NP}
\email{njhouk@gmail.com}

\title{The Mathematical Assertion Delay (MAD) Paradox: \\
A Framework for Probabilistic Truth in Computational Complexity}

\author{Nathaniel J. Houk\\
\textit{Independent Researcher}\\
\href{mailto:njhouk@gmail.com}{njhouk@gmail.com}}

\date{\today}

\begin{document}
\maketitle

\begin{abstract}
This paper introduces the Mathematical Assertion Delay (MAD) Paradox, a novel framework that challenges traditional notions of mathematical truth through the lens of temporal persistence and probabilistic verification. Building on established concepts in complexity theory and cryptographic timestamping, we formalize how mathematical statements can transition from unverified to probabilistically true through temporal persistence without contradiction. Our framework provides new insights into the nature of mathematical truth, particularly in the context of computationally intractable problems. We demonstrate applications to open problems in complexity theory and propose a blockchain-based verification mechanism that enables empirical validation of mathematical claims through economic incentives. The framework is extended to analyze multi-oracle competition dynamics and phase transitions in mathematical consensus, revealing deep connections to quantum mechanics and information theory.
\end{abstract}

[Rest of the comprehensive content we developed...]

\section*{Acknowledgments}
The author thanks the Bitcoin miners for securing the initial proof timestamp, and the mathematical community for valuable discussions.

\begin{thebibliography}{99}
\bibitem{cook1971} Cook, S. A. (1971). The complexity of theorem-proving procedures. In Proceedings of the third annual ACM symposium on Theory of computing (pp. 151-158).

\bibitem{godel1931} Gödel, K. (1931). Über formal unentscheidbare Sätze der Principia Mathematica und verwandter Systeme I. Monatshefte für mathematik und physik, 38(1), 173-198.

\bibitem{houk2025} Houk, N. J. (2025). Timeproof: A Protocol for Probabilistic Verification via Blockchain Timestamps.

\bibitem{deutsch1985} Deutsch, D. (1985). Quantum theory, the Church-Turing principle and the universal quantum computer. Proceedings of the Royal Society of London, A400, 97-117.

[Additional references...]

\end{thebibliography}
\end{document}